\selectlanguage{english}

%\textit{In this doctoral dissertation, we investigate automated obfuscation attacks that target software plagiarism detection systems.
%To that end, we propose defense mechanisms to enhance the resilience of state-of-the-art software plagiarism detectors against a broad spectrum of obfuscation attack types}.
\Abstract{%
%
Plagiarism is a prevalent challenge in computer science education, especially in introductory programming courses. Educators rely on detection systems to tackle plagiarism at scale. However, state-of-the-art systems remain vulnerable to specific obfuscation techniques that alter the structure of a program while maintaining its behavior to evade detection. Automated obfuscation attacks exacerbate this problem, particularly with recent advancements in artificial intelligence that have made automated obfuscation more accessible. Furthermore, these detection systems do not apply to modeling assignments, highlighting the need for obfuscation-resilient plagiarism detection for both programming and modeling languages.

To address these challenges, in this dissertation, we enhance state-of-the-art software plagiarism detection systems with resilience against automated obfuscation attacks. To that end, we present three key contributions. First, we propose a comprehensive threat model for obfuscation attacks on software plagiarism detection systems, examining how such attacks disrupt detection by targeting the internal program representation of detection systems. Second, we outline an approach that enables token-based plagiarism detection for artifacts of modeling assignments, applying a well-established concept to modeling education. Third, we present three novel defense mechanisms against automated obfuscation attacks that can be integrated into state-of-the-art detection systems, including attack-specific mechanisms for targeted defense and attack-independent mechanisms for broad resilience.

An empirical evaluation demonstrates the effectiveness of these contributions across real-world datasets, including programming and modeling assignments, analyzing more than four million data points. Nine different obfuscation techniques, including algorithmic and AI-based obfuscation, are employed for this evaluation. The results show that the defense mechanisms significantly improve obfuscation resilience against all nine types of attacks compared to state-of-the-art methods and, in some cases, provide complete immunity.
%
These results demonstrate not only the feasibility and practicality of these contributions in addressing the growing challenges of automated obfuscation but also their capability to enable resilient software plagiarism detection for programming and modeling assignments. This dissertation equips educators with methods to address the emerging threats of automated obfuscation attacks. Integrating these contributions into a widely used detection system allows reliable software plagiarism detection in practice.
%
}

%\textit{In dieser Dissertation beschäftigt sich mich mit automatisierten Obfuskationsangriffen auf Software-Plagiatsdetektoren.
%Hierbei stellen wir Verteidigungsmechanismen vor, welche es erlauben, geläufige Software-Plagiatsdetektoren resilient gegen verschieden Angriffsklassen zu machen
\selectlanguage{ngerman}
\Abstract[Zusammenfassung]{
%
Plagiarismus stellt eine signifikante Herausforderung in der Informatikausbildung dar und kann insbesondere in Lehrveranstaltungen zur Programmierung zum Problem werden.
Um Plagiate in großen Lehrveranstaltungen identifizieren zu können, sind Lehrende auf automatisierte Erkennungssysteme angewiesen.
Bisherige Erkennungssysteme sind jedoch anfällig für spezifische Verschleierungstechniken, welche die Programmstruktur, aber nicht das Verhalten verändern, um eine Erkennung zu umgehen.
Die Verwendung automatisierter Verschleierungstechniken verschärft dieses Problem, und speziell die jüngsten Fortschritte im Bereich der künstlichen Intelligenz machen solche Verschleierungstechniken zunehmend zugänglicher. 
Neben Programmierung ist auch die Modellierung ein wichtiger Teil der Informatikausbildung, jedoch sind die derzeitigen Erkennungssysteme nicht auf Modellierungsaufgaben anwendbar.
Dies unterstreicht den Bedarf an verschleierungsresilienter Plagiatserkennung für Programmier- und Modellierungssprachen.

Um diese Herausforderungen zu bewältigen, präsentiert die vorliegende Dissertation Erweiterungen für moderne Software-Plagiatserkennungssysteme, um ihre Resilienz gegen automatisierte Verschleierungsangriffe zu verbessern. 
Dies umfasst drei wesentliche Beiträge:
Zunächst wird ein umfassendes Bedrohungsmodell für Verschleierungsangriffe auf Plagiatserkennungssysteme definiert, um zu analysieren, wie solche Angriffe die interne Programmrepräsentation manipulieren und somit die Plagiatserkennung beeinträchtigen.
Der zweite Beitrag ermöglicht Token-basierte Plagiatserkennung für Artefakte von Modellierungsaufgaben und erweitert dabei etablierte Methoden, um sie für diese Artefakte anwendbar zu machen. 
Mit dem dritten Beitrag werden drei neuartige Abwehrmechanismen gegen automatisierte Verschleierungsangriffe vorgestellt, welche in bestehende Erkennungssysteme integriert werden können.
Die vorgestellten Mechanismen umfassen angriffsspezifische Maßnahmen zur gezielten Verteidigung gegen bestimmte Angriffe und angriffsunabhängige Ansätze für eine breite Resilienz.

Die empirische Evaluation demonstriert die Wirksamkeit der vorgestellten Beiträge anhand realer Datensätze und analysiert in diesem Rahmen über vier Millionen Datenpunkte.
Für die Evaluation werden neun verschiedene Verschleierungsangriffe eingesetzt, darunter sowohl algorithmische als auch KI-basierte.
Die Ergebnisse zeigen, dass die vorgestellten Abwehrmechanismen die Resilienz gegen alle neun Angriffstypen signifikant verbessern und in einigen Fällen vollständige Immunität gewährleisten.
Die Beiträge dieser Dissertation ermöglichen somit eine resiliente Software-Plagiatserkennung für Programmier- und Modellierungsaufgaben und erlauben Lehrenden, die wachsenden Bedrohungen durch automatisierte Verschleierungsangriffe zu bewältigen.
Die Integration dieser Beiträge in ein weitverbreitetes Erkennungssystem gewährleistet die zuverlässige Erkennung von Software-Plagiaten in der Praxis.
%
}


\endinput
